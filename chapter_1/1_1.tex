\documentclass{article}
\usepackage{amsmath}
\usepackage{amssymb}
\usepackage{enumitem}

\title{Solutions to Section 1.1 \\ Variables}
\author{bosnianbyte}
\date{February 2025}

\begin{document}

\maketitle

\section{Exercise Set 1.1}

\begin{enumerate}[label=\textbf{\arabic*.}]
\item % Solution to problem 1
\item[a.] Is there a real number $x$ such that $x^2 = -1$?
\item[b.] Does there exist a real number $x$ such that $x^2 = -1$? 
\item % Solution to problem 2
\item[a.] Is there an integer $n$ such that $n$ has a remainder of 2 when it is divided by 5 and a reminder of 3 when it is divided by 6?
\item[b.] Does there exist an integer n such that if $n$ is divided by 5 the reminder is 2 and if $n$ is divided by 6 the reminder is 3?
\item % Solution to problem 3
\item[a.] Given any two distinct real numbers $a$ and $b$, there is a real number $c$ such that $c$ is between $a$ and $b$.
\item[b.] For any two distinct real numbers $a$ and $b$, there is a real number $c$ such that $c$ is between $a$ and $b$.
\item % Solution to problem 4
\item[a.] Given any real number r, there is a real number s such that s is greater than r.
\item[b.] For any real number r, there is a real number s such that $s > r$
\item % Solution to problem 5
\item[a.] Given any positive real number r, the reciprocal of r is positive.
\item[b.] For any real number r, if r is positive, then its reciprocal is positive.
\item[c.] If a real number r is positive, then the reciprocal of r is positive.
\item % Solution to problem 6
\item[a.] Given any negative real number s, the cube root of s is negative.
\item[b.] For any real number s, if s is negative then the cube root of s is negative.
\item[c.] If a real number s is negative, then the cube root of s is negative.
\item % Solution to problem 7
\item[a.] False: There are two real numbers such that their sum is less than their difference.
\item[b.] False: There is a real number whose greater than its square.
\item[c.] True: If any integer is positive, it is less than its square.
\item[d.] True: The absolute value of the sum of two real numbers is less than or equal to the sum of the absolute value of two real numbers.
\item % Solution to problem 8
\item[a.] All squares have four sides.
\item[b.] Every square has four sides.
\item[c.] If an object is a square, then it has four sides.
\item[d.] If J is a square, then J had four sides.
\item[e.] For every square J, J has four sides.
\item % Solution to problem 9
\item[a.] All quadratic equations have at most two real solutions.
\item[b.] Every quadratic equation has at most two real solutions.
\item[c.] If an equation is quadratic, then it has at most two real solutions.
\item[d.] If E is a quadratic equation, then E has at most two real solutions.
\item[e.] For every quadratic equation E, E has at most two real solutions.
\item % Solution to problem 10
\item[a.] All nonzero real numbers have a reciprocal.
\item[b.] For every nonzero real number r, there is a reciprocal for r.
\item[c.]  For every nonzero real number r, there is a real number s such that s is a reciprocal of r.
\item % Solution to problem 11
\item[a.] All positive numbers have a positive square root.
\item[b.] For every positive number e, there is a positive square root for e.
\item[c.] For every positive number e, there is a positive number r such that r is the square root of e.
\item % Solution to problem 12
\item[a.] Some real number has the property that its product with every number leaves the number unchanged.
\item[b.] There is a real number r such that the product of r and every number leaves the number unchanged.
\item[c.] There is a real number r with the property that for every real number s, the product of r and s leaves s unchanged.
\item % Solution to problem 13
\item[a.] Some real number has the property that its product with every real number equals zero.
\item[b.] There is a real number a such that the product of a and every real number equals zero.
\item[c.] There is a real number a with the property that for every real number b, the product of a and b is zero.

\end{enumerate}

\end{document}
