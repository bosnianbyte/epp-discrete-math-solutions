\documentclass{article}
\usepackage{amsmath}
\usepackage{amssymb}
\usepackage{enumitem}

\title{Solutions to Section 1.1 \\ Logical Form and Logical Equivalence}
\author{bosnianbyte}
\date{February 2025}

\begin{document}

\maketitle

\section{Exercise Set 1.1}

\begin{enumerate}[label=\textbf{\arabic*.}]
\item % Solution to problem 1
If all algebraic expressions can be written in prefix notation, then $(a + 2b)(a^2 - b)$ can be written in prefix notation.

All algebraic expressions can be written in prefix notation.

Therefore $(a + 2b)(a^2 - b)$ can be written in prefix notation.
\item % Solution to problem 2
If all prime numbers are odd, then 2 is odd.

2 is not odd.

Therefore, it is not the case that all prime numbers are odd.
\item % Solution to problem 3
My mind is shot or logic is confusing.

My mind is not shot.

Therefore, logic is confusing.
\item % Solution to problem 4

\item % Solution to problem 5
\item[a.] Statement
\item[b.] Statement
\item[c.] Statement
\item[d.] Not a statement
\item % Solution to problem 6
\item[a.] $s \land i$
\item[b.] $\neg s \land \neg i$
\item % Solution to problem 7
$m \land \neg c$
\item % Solution to problem 8
\item[a.] $(h \land w) \land \neg s$
\item[b.] $\neg h \land (w \land s)$
\item[c.] $\neg h \land \neg w \land \neg s$
\item[d.] $(\neg w \land \neg s) \land h$
\item[e.] $h \land (\neg w \land \neg s)$
\item % Solution to problem 9
$(n \lor k) \land \neg (n \land k)$
\item % Solution to problem 10
\item[a.] $p \land q \land r$
\item[b.] $p \land \neg q$
\item[c.] $p \land (\neg q \lor \neg r)$
\item[d.] $(\neg p \land q) \land \neg r$
\item[e.] $\neg p \lor (q \land r)$
\item % Solution to problem 11
Inclusive OR
\item % Solution to problem 12
"United States President" AND (14th OR fourteenth) AND NOT amendment
\item % Solution to problem 13
(jaguar AND cheetah) AND (speed or fastest) AND NOT (car or automobile or auto)
\item % Solution to problem 14
\[
\begin{array}{|c|c||c||c|}
\hline
p & q & \neg p & \neg p \land q\\
\hline
T & T & F & F \\
T & F & F & F \\
F & T & T & T \\
F & F & T & F \\
\hline
\end{array}
\]
\item % Solution to problem 15
\[
\begin{array}{|c|c||c|c||c|}
\hline
p & q & \neg (p \land q) & p \lor q & \neg (p \land q) \lor (p \lor q)\\
\hline
T & T & F & T & T \\
T & F & T & T & T \\
F & T & T & T & T \\
F & F & T & F & T \\
\hline
\end{array}
\]
\item % Solution to problem 16
\[
\begin{array}{|c|c|c||c||c|}
\hline
p & q & r & q \land r & p \land (q \land r)\\
\hline
T & T & T & T & T \\
T & T & F & F & F \\
T & F & T & F & F \\
T & F & F & F & F \\
F & T & T & T & F \\
F & T & F & F & F \\
F & F & T & F & F \\
F & F & F & F & F \\
\hline
\end{array}
\]
\item % Solution to problem 17
\[
\begin{array}{|c|c|c||c||c||c|}
\hline
p & q & r & \neg q & \neg q \lor r & p \land (\neg q \lor r)\\
\hline
T & T & T & F & T & T \\
T & T & F & F & F & F \\
T & F & T & T & T & T \\
T & F & F & T & T & T \\
F & T & T & F & T & F \\
F & T & F & F & F & F \\
F & F & T & T & T & F \\
F & F & F & T & T & F \\
\hline
\end{array}
\]
\item % Solution to problem 18
\[
\begin{array}{|c|c|c||c|c||c|c||c|c||c|}
\hline
p & q & r & \neg p & \neg r & \neg p \lor q & q \land \neg r & p \lor (\neg p \lor q) & \neg (q \lor \neg r) & (p \lor (\neg p \lor q)) \land \neg(q \lor \neg r)\\
\hline
T & T & T & F & F & T & F & T & F & F \\
T & T & F & F & T & T & T & T & F & F \\
T & F & T & F & F & F & F & T & T & T \\
T & F & F & F & T & F & F & T & F & F \\
F & T & T & T & F & T & F & T & F & F \\
F & T & F & T & T & T & T & T & F & F \\
F & F & T & T & F & T & F & T & T & T \\
F & F & F & T & T & T & F & T & F & F \\
\hline
\end{array}
\]
\item % Solution to problem 19
The statements 'p' and '$p \lor (p \land q)$' always have the same truth values, so they are logically equivalent.
\[
\begin{array}{|c|c||c||c|}
\hline
p & q & p \land q & p \lor (p \land q)\\
\hline
T & T & T & T \\
T & F & F & T \\
F & T & F & F \\
F & F & F & F \\
\hline
\end{array}
\]
\item % Solution to problem 20
The statements '$\neg (p \land q)$' and '$\neg p \land \neg q$' do not have the same truth values, so they aren't logically equivalent.
\[
\begin{array}{|c|c||c|c||c||c|c|}
\hline
p & q & \neg p & \neg q & p \land q & \neg (p \land q) & \neg p \land \neg q\\
\hline
T & T & F & F & T & F & F \\
T & F & F & T & F & T & F \\
F & T & T & F & F & T & F \\
F & F & T & T & F & T & T \\
\hline 
\end{array}
\]
\item % Solution to problem 21
The statements '$p \lor t$' and 't' always have the same truth values, so they are logically equivalent.
\[
\begin{array}{|c|c||c|}
\hline
p & t & p \lor t\\
\hline
T & T & T \\
T & T & T \\
F & T & T \\
F & T & T \\
\hline 
\end{array}
\]
\item % Solution to problem 22
The statements '$p \land t$' and 't' do not have the same truth values, so they aren't logically equivalent.
\[
\begin{array}{|c|c||c|}
\hline
p & t & p \land t\\
\hline
T & T & T \\
T & T & T \\
F & T & F \\
F & T & F \\
\hline 
\end{array}
\]
\item % Solution to problem 23
The statements '$(p \land q) \land r$' and '$p \land (q \land r)$' have the same truth values, so they are logically equivalent.
\[
\begin{array}{|c|c|c||c|c|}
\hline
p & q & r & (p \land q) \land r & p \land (q \land r)\\
\hline
T & T & T & T & T \\
T & T & F & F & F \\
T & F & T & F & F \\
T & F & F & F & F \\
F & T & T & F & F \\
F & T & F & F & F \\
F & F & T & F & F \\
F & F & F & F & F \\
\hline 
\end{array}
\]
\item % Solution to problem 24
The statement '$p \land (q \lor r)$' and '$(p \land q) \lor (p \land r)$' have the same truth values, so they are logically equivalent.
\[
\begin{array}{|c|c|c||c|c|c||c|c|}
\hline
p & q & r & q \lor r & p \land q & p \land r & p \land (q \lor r) & (p \land q) \lor (p \land r)\\
\hline
T & T & T & T & T & T & T & T \\
T & T & F & T & T & F & T & T \\
T & F & T & T & F & T & T & T \\
T & F & F & F & F & F & F & F \\
F & T & T & T & F & F & F & F \\
F & T & F & T & F & F & F & F \\
F & F & T & T & F & F & F & F \\
F & F & F & F & F & F & F & F \\
\hline 
\end{array}
\]
\item % Solution to problem 25
The statements '$(p \land q) \lor r$' and '$p \land (q \lor r)$' have different truth values, so they aren't logically equivalent.
\[
\begin{array}{|c|c|c||c|c|c||c|c|}
\hline
p & q & r & q \lor r & p \land q & (p \land q) \lor r & p \land (q \lor r) \\
\hline
T & T & T & T & T & T & T \\
T & T & F & T & T & T & T \\
T & F & T & T & F & T & T \\
T & F & F & F & F & F & F \\
F & T & T & T & F & T & F \\
F & T & F & T & F & F & F \\
F & F & T & T & F & T & F \\
F & F & F & F & F & F & F \\
\hline 
\end{array}
\]
\item % Solution to problem 26
The statements '$(p \lor q) \lor (p \land r)$' and '$(p \lor q) \land r$' have the different truth values, so they aren't logically equivalent.
\[
\begin{array}{|c|c|c||c|c||c|c|}
\hline
p & q & r & p \lor q & p \land r & (p \lor q) \lor (p \land r) & (p \lor q) \land r\\
\hline
T & T & T & T & T & T & T \\
T & T & F & T & F & T & F \\
T & F & T & T & T & T & T \\
T & F & F & T & F & T & F \\
F & T & T & T & F & T & T \\
F & T & F & T & F & T & F \\
F & F & T & F & F & F & F \\
F & F & F & F & F & F & F \\
\hline 
\end{array}
\]
\item % Solution to problem 27
The statements '$((\neg p \lor q) \land (p \lor \neg r)) \land (\neg p \lor \neg q)$' and '$\neg (p \lor r)$' have the same truth values, so they are logically equivalent.
\[
\begin{array}{|c|c|c||c|c|c||c|c|c||c|c||c|}
\hline
p & q & r & \neg p & \neg q & \neg r & \neg p \lor q & p \lor \neg r & \neg p \lor \neg q & p \lor r\\
\hline
T & T & T & F & F & F & T & T & F & T \\
T & T & F & F & F & T & T & T & F & T \\
T & F & T & F & T & F & F & T & T & T \\
T & F & F & F & T & T & F & T & T & T \\
F & T & T & T & F & F & T & F & T & T \\
F & T & F & T & F & T & T & T & T & F \\
F & F & T & T & T & F & T & F & T & T \\
F & F & F & T & T & T & T & T & T & F \\
\hline 
\end{array}
\]
\[
\begin{array}{|c|c|}
\hline
\neg (p \lor r) & ((\neg p \lor q) \land (p \lor \neg r)) \land (\neg p \lor \neg q)\\
\hline
F & F \\
F & F \\
F & F \\
F & F \\
F & F \\
T & T \\
F & F \\
T & T \\
\hline 
\end{array}
\]
\item % Solution to problem 28
The statements '$(r \lor p) \land ((\neg r \lor (p \land q )) \land (r \lor q))$' and '$p \land q$' have the same truth values, so they are logically equivalent.
\[
\begin{array}{|c|c|c||c||c|c|c|c||c|c|}
\hline
p & q & r & \neg r & r \lor p & p \land q & \neg r \lor (p \land q) & r \lor q & p \land q & (r \lor p) \land ((\neg r \lor (p \land q)) \land (r \lor q))\\
\hline
T & T & T & F & T & T & T & T & T & T \\
T & T & F & T & T & T & T & T & T & T \\
T & F & T & F & T & F & F & T & F & F \\
T & F & F & T & T & F & T & F & F & F \\
F & T & T & F & T & F & F & T & F & F \\
F & T & F & T & F & F & T & T & F & F \\
F & F & T & F & T & F & F & T & F & F \\
F & F & F & T & F & F & T & F & F & F \\
\hline 
\end{array}
\]
\item % Solution to problem 29
Hal is not a math major or Hal's sister is not a computer science major.
\item % Solution to problem 30
Sam is not an orange belt or Kate is not a red belt.
\item % Solution to problem 31
The connector is not loose and the machine is not unplugged
\item % Solution to problem 32
This computer program does not have a logical error in the first ten lines and it is not being run with an incomplete data set
\item % Solution to problem 33
The dollar is not at an all-time high or the stock market is not at a record low.
\item % Solution to problem 34
The train is not late and my watch is not fast.
\item % Solution to problem 35
$-2 \geq x \lor x \geq 7$
\item % Solution to problem 36
$-10 \geq x \lor x \geq 2$
\item % Solution to problem 37
$1 \leq x \lor x < -3$
\item % Solution to problem 38
$0 \leq x \lor x < -7$
\item % Solution to problem 39
$(num\_orders \leq 100 \lor num\_instock > 500) \land num\_instock \geq 200$
\item % Solution to problem 40
$(num\_orders \geq 50 \lor num\_instock \leq 300) \land (50 > num\_orders \geq 75 \lor num\_instock \leq 500)$
\item % Solution to problem 41
The statement '$(p \land q) \lor (\neg p \lor (p \land \neg q))$' has all T truth values, so it is a tautology.
\[
\begin{array}{|c|c||c|c||c|c|c|}
\hline
p & q & \neg p & \neg q & p \land q & p \land \neg q & (p \land q) \lor (\neg p \lor (p \land \neg q))\\
\hline
T & T & F & F & T & F & T \\
T & F & F & T & F & T & T \\
F & T & T & F & F & F & T \\
F & F & T & T & F & F & T \\
\hline 
\end{array}
\]
\item % Solution to problem 42
The statement '$(p \land \neg q) \land (\neg p \lor q)$' has all F truth values, so it is a contradiction.
\[
\begin{array}{|c|c||c|c||c|c|c|}
\hline
p & q & \neg p & \neg q & p \land \neg q & \neg p \lor q &(p \land \neg q) \land (\neg p \lor q)\\
\hline
T & T & F & F & F & T & F \\
T & F & F & T & T & F & F \\
F & T & T & F & F & T & F \\
F & F & T & T & F & T & F \\
\hline 
\end{array}
\]
\item % Solution to problem 43
The statement '$((\neg p \land q) \land (q \land r)) \land \neg q$' has all F truth values, so it is a contradiction.
\[
\begin{array}{|c|c|c||c|c||c|c|c||c|}
\hline
p & q & r & \neg p & \neg q & p \land q & \neg p \land q & q \land r & ((\neg p \land q) \land (q \land r)) \land \neg q\\
\hline
T & T & T & F & F & T & F & T & F \\
T & T & F & F & F & T & F & F & F \\
T & F & T & F & T & F & F & F & F \\
T & F & F & F & T & F & F & F & F \\
F & T & T & T & F & F & T & T & F \\
F & T & F & T & F & F & T & F & F \\
F & F & T & T & T & F & F & F & F \\
F & F & F & T & T & F & F & F & F \\
\hline 
\end{array}
\]
\item % Solution to problem 44
The statement '$(\neg p \lor q) \lor (p \land \neg q)$' has all T truth values, so it is a tautology.
\[
\begin{array}{|c|c||c|c||c|c|c|c||c|}
\hline
p & q & \neg p & \neg q & p \lor q & p \land q & \neg p \lor q & p \land \neg q & (\neg p \lor q) \lor (p \land \neg q)\\
\hline
T & T & F & F & T & T & T & F & T \\
T & F & F & T & T & F & F & T & T \\
F & T & T & F & T & F & T & F & T \\
F & F & T & T & F & F & T & F & T \\
\hline 
\end{array}
\]
\item % Solution to problem 45
\begin{tabbing}
$(p \land \neg q) \lor (p \land q)$ \= $\equiv p \lor (\neg q \lor q)$ \hspace{2.8cm} \= a. Distributive law\\
\> $\equiv p \land (q \lor \neg q)$ \> b. Commutative law\\
\> $\equiv p \land t$ \> c. Negation law\\
\> $\equiv p$ \> d. Identity law
\end{tabbing}
\item % Solution to problem 46
\begin{tabbing}
$(p \lor \neg q) \land (\neg p \lor \neg q)$ \= $\equiv (\neg q \lor p) \land (\neg q \lor \neg p)$ \hspace{1cm} \= a. Commutative law\\
\> $\equiv \neg q \lor (p \land \neg p)$ \> b. Distributive law\\
\> $\equiv \neg q \lor c$ \> c. Negation law\\
\> $\equiv \neg q$ \> d. Identity law
\end{tabbing}
\item % Solution to problem 47
$(p \land \neg q) \lor p \equiv p$

1. Absorption law
\item % Solution to problem 48
$p \land (\neg q \lor p) \equiv p$

1. Absorption law
\item % Solution to problem 49
$\neg (p \lor \neg q) \lor (\neg p \land \neg q) \equiv \neg p$

1. De Morgan's law: $(\neg p \land q) \lor (\neg p \land \neg q)$

2. Distributive law: $\neg p \land (q \lor \neg q)$

3. Negation law: $\neg p \land t$

4. Identity law: $\neg p$
\item % Solution to problem 50
$\neg ((\neg p \land q) \lor (\neg p \land \neg q)) \lor (p \land q) \equiv p$

1. Distributive law: $\neg (\neg p \land (q \lor \neg q)) \lor (p \land q)$

2. Negation law: $\neg (\neg p \land t) \lor (p \land q)$

3. Identity law: $\neg (\neg p) \lor (p \land q)$

4. Double negative law: $p \lor (p \land q)$

5. Absorption law: $p$
\item % Solution to problem 51
$(p \land (\neg(\neg p \lor q))) \lor (p \land q) \equiv p$

1. De Morgan's law: $(p \land (p \land \neg q)) \lor (p \land q)$

2. Associative law: $((p \land p) \land \neg q) \lor (p \land q)$ 

3. Idempotent law: $(p \land \neg q) \lor (p \land q)$

4. Distributive law: $p \land (\neg q \lor q)$

5. Negation law: $p \land t$

6. Identity law: $p$
\item % Solution to problem 52
a. $p \oplus p \equiv c$

b. Yes because of the Associative laws.

c. Yes because of the Distributive laws.
\item % Solution to problem 53
Sarcasm is an example of a double positive being equivalent to a negative in standard English.
\item % Solution to problem 54
$p \lor (q \land (\neg r \lor (\neg s \lor \neg t)))$

\end{enumerate}

\end{document}
